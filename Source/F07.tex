\subsection*{Solution to Fall 2007, \#1}\label{f071}
We have
\begin{align*}
u(x, t) &= \frac{1}{\sqrt{4\pi t}}\int_{-\infty}^{\infty}\phi(y)e^{-\frac{(x - y)^{2}}{4t}}\, dy = \frac{1}{\sqrt{4\pi t}}\int_{-\infty}^{\infty}\phi(y)e^{-(\frac{x - y}{\sqrt{4t}})^{2}}\, dy\\
& = -\frac{1}{\sqrt{4\pi t}}\int_{\infty}^{-\infty}\phi(x - u\sqrt{4t})e^{-u^{2}}\sqrt{4t}\, du = \frac{1}{\sqrt{\pi}}\int_{-\infty}^{\infty}\phi(x - u\sqrt{4t})e^{-u^{2}}\, du
\end{align*}
where the third equality is by the change of variables $u = (x - y)/\sqrt{4t}$. Since $\phi$ is bounded and $\phi \rightarrow \phi_{0}$ as $|x| \rightarrow \infty$,
by the Dominated Convergence theorem, for each fixed $x$,
\begin{align*}
\lim_{t \rightarrow \infty}u(x, t) &= \lim_{t \rightarrow \infty}\frac{1}{\sqrt{\pi}}\int_{-\infty}^{\infty}\phi(x - u\sqrt{4t})e^{-u^{2}}\, du\\
& = \frac{1}{\sqrt{\pi}}\int_{-\infty}^{\infty}(\lim_{t \rightarrow \infty}\phi(x - u\sqrt{4t}))e^{-u^{2}}\, du = \phi_{0}\cdot \frac{1}{\sqrt{\pi}}\int_{-\infty}^{\infty}e^{-u^{2}}\, du = \phi_{0}.
\end{align*}
\hfill\qed

\subsection*{Solution to Fall 2007, \#2}\label{f072}
We want to show that if $u, v$ smooth with
\begin{align*}
\Delta u + \abn{\nabla u}^{2} = \Delta v + \abn{\nabla v}^{2} & \text{ in } \Om\\
u = v & \text{ on } \pr \Om
\end{align*}
then $u = v$ in $\Om$. Let $w = u - v$. Then
\begin{align*}
\lap w + \abn{\del u}^{2} - \abn{\del v}^{2} = 0 & \text{ in } \Om\\
w = 0 & \text{ on } \pr \Om.
\end{align*}
Note
$$\abn{\del u}^{2} - \abn{\del v}^{2} = (\del u - \del v) \cdot (\del u + \del v) = \del w \cdot (\del u + \del v).$$
Let $y = w + \vep e^{\ld x_{1}}$. Then
$$\del y = \del w + (\vep\ld e^{\ld x_{1}}, 0, \ldots, 0)$$
and $\lap y = \lap w + \vep\ld^{2}e^{\ld x_{1}}$.
Thus
$$\del y \cdot (\del u + \del v) = \del w \cdot (\del u + \del v) + \vep \ld e^{\ld x_{1}}(u_{x_{1}} + v_{x_{1}})$$
and
$$\lap y + \del y \cdot (\del u + \del v) = \vep\ld^{2}e^{\ld x_{1}} + \vep \ld e^{\ld x_{1}}(u_{x_{1}} + v_{x_{1}}) = \vep e^{\ld x_{1}}(\ld^{2} + \ld(u_{x_{1}} + v_{x_{1}}))$$
since $u$, $v$ are smooth and $\Om$ is bounded, $u_{x_{1}} + v_{x_{1}}$ is bounded on $\Om$. Thus choose
$\ld$ sufficiently large such that $\ld^{2} + \ld(u_{x_1} + v_{x_1}) > 0$ on $\Om$.
Then $\lap y + \del y \cdot (\del u + \del v) > 0$ on $\Om$.
With this choice of $\ld$, we claim
$\max{\ov{\Om}}y = \max_{\pr\Om}y$. Suppose $x_{0}$ was such that $x_0 \in \Om$ and $y(x_{0}) = \max_{\ov{\Om}}y$.
Then at $x_{0}$,
$$\lap y + \del y \cdot (\del u + \del v) \leq 0,$$
a contradiction. Therefore $$\max_{\ov{\Om}}y = \max_{\pr\Om}y.$$
We then have
$$0 = \max_{\pr\Om}w \leq \max_{\ov{\Om}}w \leq \max_{\ov{\Om}}y = \max_{\pr\Om}y = \max_{\pr\Om}\vep e^{2\ld x_{1}} \leq C\vep$$
for some $C$ depending only on $\ov{\Om}$. Since $\vep$ was arbitrary, letting $\vep \rightarrow 0$ shows that
$w \leq 0$ on $\ov{\Om}$ which implies that $u \leq v$ on $\ov{\Om}$.
Interchanging the roles of $u, v$ above then shows $v \leq u$ on $\ov{\Om}$.
Thus $u = v$ on $\ov{\Om}$.
\hfill\qed

\subsection*{Solution to Fall 2007, \#3}\label{f073}
We have $\lap u + \ld u = 0$ in $\{0 < x < a, -\infty < y < \infty\}$ with $u(0, y) = 0$ and $u(a, y) = 0$.
Let $u(x, y) = F(x)G(y)$, these boundary conditions imply that $F(0) = 0$ and $F(a) = 0$.

We will only consider the case when $\ld = 0$. A similar argument will show the result when $\ld < 0$ or $\ld > 0$
(see the solution to Winter 2004, \#1 for the solution to the PDE in these cases). Since $u(x, y) = F(x)G(y)$ and $\ld = 0$,
$$F''(x)G(y) + F(x)G''(y) = 0$$
and hence
$$\frac{F''(x)}{F(x)} = -\frac{G''(y)}{G(y)} = -\mu$$
for some constant $\mu$. Since we want nontrivial solutions, the only such solutions occur when $\mu > 0$ (by our boundary conditions on $F$).
Then
$F''(x) + \mu F(x) = 0$ and hence
$F(x) = A\cos\sqrt{\mu}x + B\sin\sqrt{\mu}x$. Imposing the condition that $F(0) = 0$ yields that $A = 0$ and hence
$F(x) = B\sin\sqrt{\mu}x$. Since $F(a) = 0$, $B\sin\sqrt{\mu}a = 0$ and hence $\sqrt{\mu}a = n\pi$ for $n = 1, 2, \ldots$
which implies $\mu_{n} = (n\pi/a)^{2}$ and hence $F_{n}(x) = \sin(\frac{n\pi}{a}x)$.
Since $G''(y)/G(y) = \mu$, $G = Ce^{-n\pi y/a} + De^{n\pi y/a}$. Thus
$$u(x, y) = \sum_{n \geq 1}(C_{n}e^{-n\pi y/a} + D_{n}e^{n\pi y/a})\sin\frac{n\pi x}{a}.$$
If $\int_{-\infty}^{\infty}\int_{0}^{a}|u(x, y)|^{2}\, dx\, dy < \infty$, as
\begin{align*}
\int_{0}^{a}\sin\frac{n\pi x}{a}\sin\frac{m\pi x}{a}\, dx =\frac{a}{2}1_{m = n},
\end{align*}
we have
\begin{align*}
\int_{0}^{a}u(x, y)^{2}\,dx &= \int_{0}^{a}\sum_{n \geq 1}(C_{n}e^{-n\pi y/a} + D_{n}e^{n\pi y/a})^{2}(\sin \frac{n\pi x}{a})^{2}\, dx\\
& = \sum_{n \geq 1}(C_{n}e^{-n\pi y/a} + D_{n}e^{n\pi y/a})^{2}\frac{a}{2} = \frac{a}{2}\sum_{n \geq 1}C_{n}^{2}e^{-2n\pi y/a} + 2C_{n}D_{n} + D_{n}^{2}e^{2n\pi y/a}.
\end{align*}
Since $\int_{-\infty}^{\infty}e^{\pm 2n\pi y/a}\, dy = \infty$, the only way for $\nms{u}_{L^{2}(S)} < \infty$ is to have $C_{n} = D_{n} = 0$
for all $n$. Thus $u = 0$ in the case when $\ld = 0$.
\hfill\qed

\subsection*{Solution to Fall 2007, \#4}\label{s074}
Let $u, v$ be two smooth solutions. Let $w := u - v$. Then
\begin{align*}
w_{tt} + 2w_{xt} - w_{xx} + aw_{x} &= 0\\
w(x, 0) = 0, w_{t}(x, 0) &= 0.
\end{align*}
Let
$$e(t) := \frac{1}{2}\int_{\R}w_{t}^{2} + w_{x}^{2}\, dx.$$
Then
\ba
\dot{e}(t) = \int_{\R}w_{t}w_{tt} + w_{x}w_{xt}\, dx &= \int_{\R}w_{t}w_{tt} - w_{xx}w_{t}\, dx\\
& = \int_{\R}w_{t}(-2w_{xt} - aw_{x})\, dx = \int_{\R}-2w_{t}w_{xt} - aw_{x}w_{t}\, dx.
\ea
As $\int_{\R}w_{t}w_{xt}\, dx = -\int_{\R}w_{xt}w_{t}\, dx$, we have $\int_{\R}w_{t}w_{xt}\, dx = 0$
and hence
\[
\dot{e}(t) = \int_{\R}a(x, t)w_{x}w_{t}\, dx \leq \int_{\R}\abn{a(x, t)}\abn{w_{x}}\abn{w_{t}}\, dx \leq \sup |a| \int_{\R}\frac{1}{2}w_{x}^{2} + \frac{1}{2}w_{t}^{2}\, dx \leq (\sup |a|)e(t).
\]
By Gronwall's inequality, $e(t) \leq e(0)\exp((\sup|a|)t)$. Since $e(0) = 0$,
$e(t) = 0$ for all $t$. Therefore $w \equiv 0$. This proves uniqueness.
\hfill\qed

\sub{Solution to Fall 2007, \#5}\label{s075}
\ssb{Solution to $(a)$ and $(b)$}
Separating variables and solving yields that
$$u(t) = (-\alpha ct + u_{0}^{-\alpha})^{-1/\alpha} = \bigg(\frac{1}{\frac{1}{u_{0}\alpha} - \alpha ct}\bigg)^{1/\alpha}.$$
and hence the blowup time is when $1/u_{0}^{\alpha} = \alpha ct$, that is $t = 1/(\alpha cu_{0}^{\alpha})$.
\hq

\ssb{Solution to $(c)$}
Fix $c$, $u_{0}$, we want to minimize $1/(c\alpha u_{0}^{\alpha})$. Since $c > 0$, this is the same
as minimizing $1/\alpha u_{0}^{\alpha}$. This is the same as minimizing $\log(1/(\alpha u_{0}^{\alpha}))$.
Let $F(\alpha) := \log(1/(\alpha u_{0}^{\alpha})) = -\log \alpha - \alpha \log u_{0}$.
Then $F'(\alpha) = -1/\alpha - \log u_{0}$ and $F''(\alpha) = 1/\alpha^{2} \geq 0$.
Thus the critical point of $F$ is $\alpha = -1/\log u_0$ which is a minimum.
Note $0 < u_0 < 1$ and so $-1/\log u_0 > 0$. Thus the $\alpha$ that minimizes $t_{\ast}$ is $\alpha = -1/\log u_0$.
\hq

\sub{Solution to Fall 2007, \#6}\label{s076}
The trick to solving these multidimensional method of characteristics problems is to first solve the analogous 1D problem
and then try to mimic the steps for the multidimensional case.
We first solve the 1D equation
\ba
u_{t} + uu_{x} &= u\\
u(x, 0) &= x.
\ea
We have $F(p, q, z, x, t) = q + zp - z$
and hence
\ba
\dot{x} &= z  & x(0) &= x_{0}\\
\dot{t} &= 1  & t(0) &= 0\\
\dot{z} &= z  & z(0) &= x_{0}
\ea
which implies $t(s) = s$, $z(s) = x_0 e^{s}$ and $x(s) = x_{0}e^{s}$. Therefore $u(x, t) = x$.

Having solved the 1D equation, let us now solve the multidimensional equation.
We want to solve
\ba
\mb{u}_{t} + \mb{u} \cdot \del \mb{u} &= \mb{u}\\
\mb{u}(\mb{x}, 0) &= \mb{x}.
\ea
We have
\ba
\dot{\mb{x}} &= \mb{z} & \mb{x}(0) &= \mb{x_{0}}\\
\dot{t}      &= 1      & t(0)      &= 0\\
\dot{\mb{z}} &= \mb{z} & \mb{z}(0) &= \mb{x_{0}}
\ea
which implies $t(s) = s$, $\mb{z}(s) = \smat{e^{s}}{}{}{e^s}\mb{x_{0}}$, and $\mb{x}(s) = \smat{e^{s}}{}{}{e^s}\mb{x_{0}}$.
Therefore $\mb{u}(\mb{x}, t) = \mb{x}$.
\hq

\sub{Solution to Fall 2007, \#7}\label{s077}
\ssb{Solution to $(a)$}
We have
\ba
-\ld\int_{0}^{L}u^{2}\, dx &= \int_{0}^{L}(u'' - au)u\, dx = \int_{0}^{L}u''u\, dx - \int_{0}^{L}au^{2}\, dx\\
& = -\int_{0}^{L}u'^{2}\, dx - \int_{0}^{L}au^{2}\, dx \leq -\min_{x \in [0, L]}a\int_{0}^{L}u^{2}\, dx < 0
\ea
where the last inequality we have used that $a > 0$ and that $\int_{0}^{L}u^{2}\, dx \neq 0$ since otherwise this would imply that $u = 0$.
Therefore $\ld > 0$.
\hq

\ssb{Solution to $(b)$}
Let $a(x) = -1$, $L = 2\pi$. Then $(\sin x)'' + (\sin x) = 0\cdot \sin x$. Thus $a < 0$ does not imply $\ld < 0$.
\hq

\ssb{Solution to $(c)$}
The argument in this part is similar to that of the one given in Fall 2003, \#2.
The operator $Tu = u'' - a(x)u$ is Sturm-Liouville (see the review at the end of the solutions).
The smallest eigenvalue is given by
$$\ld_{L} = \min_{\st{u \in H^{1}_{0}([0, L])\\u \not\equiv 0}}-\frac{\ips{u, Tu}}{\ips{u, u}}$$
and hence
$$-\ld_{L} = \max_{\st{u \in H_{0}^{1}([0, L])\\u \not\equiv 0}}\frac{\ips{u, Tu}}{\ips{u, u}}.$$
We will now show that $f(L) = \max_{\st{u \in H_{0}^{1}([0, L])\\\\u \not\equiv 0}}\frac{\ips{u, Tu}}{\ips{u, u}}$
is (strictly!) increasing in $L$. Since $H_{0}^{1}([0, L_{1}]) \subset H_{0}^{1}([0, L_{2}])$
for $L_1 < L_2$, we have that
\ba
\max_{\st{u \in H_{0}^{1}([0, L_1])\\u \not\equiv 0}}\frac{\ips{u, Tu}}{\ips{u, u}} \leq \max_{\st{u \in H_{0}^{1}([0, L_{2}])\\u \not\equiv 0}}\frac{\ips{u, Tu}}{\ips{u, u}}.
\ea
We now show that this inequality is in fact strict which shows strict increasing of $f(L)$.
We have
\ba
\frac{\ips{u, Tu}}{\ips{u, u}} = \frac{\int_{0}^{L}u(u'' - au)\, dx}{\int_{0}^{L}u^{2}\, dx} = \frac{-\int_{0}^{L}u'^{2}\, dx}{\int_{0}^{L}u^{2}\, dx} - a
\ea
and hence
\ba
\max_{\st{u \in H_{0}^{1}([0, L])\\u \not\equiv 0}}\frac{\ips{u, Tu}}{\ips{u, u}} = \bigg(\max_{\st{u \in H_{0}^{1}([0, L])\\u \not\equiv 0}}\frac{-\int_{0}^{L}u'^{2}\, dx}{\int_{0}^{L}u^{2}\, dx}\bigg) - a.
\ea
Thus to show $f(L)$ is increasing in $L$, it suffices to show that
$$g(L) := \max_{\st{u \in H_{0}^{1}([0, L])\\u \not\equiv 0}}\frac{-\int_{0}^{L}u'^{2}\, dx}{\int_{0}^{L}u^{2}\, dx}$$
is increasing in $L$. We have
$g(L_{1}) \leq g(L_{2})$ for $L_{1} < L_{2}$ and now we show we cannot have equality. Suppose $g(L_{1}) = g(L_{2})$.
Let $$F_{L}[u] := \frac{-\int_{0}^{L}u'^{2}\, dx}{\int_{0}^{L}u^{2}\, dx}.$$ The maximum $\wt{u}$ of $F_{L}[u]$ satisfies
\ba
0 &= \lim_{\vep \rightarrow 0}\frac{1}{\vep}(F_{L}[\wt{u} + \vep v] - F_{L}[\wt{u}]) = \lim_{\vep \rightarrow 0}\frac{1}{\vep}\bigg(\frac{-\int_{0}^{L}(\wt{u}' + \vep v')^{2}\, dx}{\int_{0}^{L}(\wt{u} + \vep v)^{2}\, dx} - \frac{-\int_{0}^{L}\wt{u}'^{2}\, dx}{\int_{0}^{L}\wt{u}^{2}\, dx}\bigg)\\
&=\lim_{\vep \rightarrow 0}\frac{1}{\vep}\bigg(\frac{(-\int_{0}^{L}\wt{u}'^{2} + 2\vep \wt{u}'v' + \vep^{2}v'^{2}\, dx)\int_{0}^{L}\wt{u}^{2}\, dx + \int_{0}^{L}\wt{u}'^{2}\, dx \int_{0}^{L}\wt{u}^{2} + 2\vep \wt{u}v + \vep^{2}v^{2}\, dx}{\int_{0}^{L}(\wt{u} + \vep v)^{2}\, dx\int_{0}^{L}\wt{u}^{2}\, dx}\bigg)
\ea
for all $v \in H_{0}^{1}([0, L])$. Therefore
$$0 = (-\int_{0}^{L}\wt{u}'v'\, dx)(\int_{0}^{L}\wt{u}^{2}\, dx) + (\int_{0}^{L}\wt{u}'^{2}\, dx)(\int_{0}^{L}\wt{u}v\, dx)$$
and hence
$$0 = (\int_{0}^{L}\wt{u}^{2}\, dx)(\int_{0}^{L}\wt{u}''v\, dx) + (\int_{0}^{L}\wt{u}'^{2}\, dx)(\int_{0}^{L}\wt{u}v\, dx).$$
Let $m :=\int_{0}^{L}\wt{u}'^{2}\, dx/\int_{0}^{L}\wt{u}^{2}\, dx$. Then $\int_{0}^{L}(\wt{u}'' + m\wt{u})v\, dx = 0$ for all $v \in H_{0}^{1}([0, L])$.
Thus $\wt{u}$ is a Dirichlet eigenfunction of the Laplacian in $[0, L]$ and hence is real analytic in $[0, L]$.

Let $\wt{u}_{1}$ be the maximizer associated to $g(L_{1})$. Then extend $\wt{u}_{1}$ to be a function (which we will still call $\wt{u}_{1}$)
in $H_{0}^{1}([0, L_{2}])$ by setting $\wt{u}_{1} = 0$
on $(L_{1}, L_{2}]$. Then as we assumed that $g(L_{1}) = g(L_{2})$,
\ba
\frac{-\int_{0}^{L_{2}}\wt{u}_{1}'^{2}\, dx}{\int_{0}^{L_{2}}\wt{u}_{1}^{2}\, d} = \max_{\st{u \in H_{0}^{1}([0, L_{2}])\\u \not\equiv 0}}\frac{-\int_{0}^{L_{2}}u'^{2}\, dx}{\int_{0}^{L_{2}}u^{2}\, dx}.
\ea
Therefore $\wt{u}_{1}$ is a Dirichlet eigenfunction of $\Delta$ in $[0, L_{2}]$ and hence
is real analytic. But $\wt{u}_{1}= 0$ on $(L_{1}, L_{2}]$ and hence $\wt{u}_{1} = 0$ in all of $[0, L_{2}]$ (since
if a real analytic function vanishes on an open set it vanishes everywhere it is real analytic), a contradiction.
Therefore we cannot have $g(L_{1}) = g(L_{2})$ and so $g(L_{1}) < g(L_{2})$. Therefore $\ld_{L}$ is a strictly decreasing function of $L$.
\hq

\sub{Solution to Fall 2007, \#8}\label{s078}
\ssb{Solution to $(a)$}
By the Maximum Principle for the heat equation
\ba
\min_{\ov{\Om_{i}(t)} \times [0, T]}u_{i} = \min_{(\ov{\Om_{i}(t)} \times [0, T]) \bs (\Om_{i}(t) \times (0, T])}u_{i} = 0.
\ea
Now suppose $u_{i}(x_0, t_0) = 0$ for some $x_0 \in \Om_{i}(t_0)$, $0 < t_0 \leq T$.
Then by the Maximum Principle, $u_i \equiv 0$ everywhere, but this contradicts that
$u_{i}(x, 0) = f(x) > 0$. Thus $u_{i} > 0$ for all $x \in \Om_{i}(t)$ and $0 < t \leq T$.
\hq

\ssb{Solution to $(b)$}
Let $w:= u_{2} - u_{1}$. Then
$w_{t} - \lap w = 0$ for $x \in \Om_{1}(t)$, $0 \leq t \leq T$.
Note that $w(x, 0) = 0$ for $x \in \Om_{1}(0)$ and for $x \in \pr\Om_{1}(t)$,
$w(x, t) = u_{2}(x, t) - u_{1}(x, t) = u_{2}(x, t) > 0$
where the inequality and second equality is because $\pr\Om_{1} \subset \Om_{2}$ and part $(a)$.
Then by the same proof as in part $(a)$, we have $w > 0$ for all $x \in \Om_{1}(t)$ and $0 < t \leq T$.
\hq
